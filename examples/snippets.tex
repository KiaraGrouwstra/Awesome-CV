% NL:

% intro
Ik ben een leergierige ICT’er met ervaring in business intelligence, big data, front-end engineering, functional programming, compilers, en machine learning. Ik ben tevens actief in de open-source community (Github), en probeer in m’n vrije tijd verder te leren over het bouwen van neural networks.

% excel
Tien jaar terug vertelde m’n vader dat hij nog niemand was tegengekomen die beter met Excel was dan hij. “Tijd om daar wat aan te doen”, besloot ik. Ik heb sindsdien verscheidene Excel projecten gedaan, o.a. om info van het Chinese social media platform Weibo te data-minen om hier inzichten uit te halen voor multinationals in Shanghai.

% bi
Ik ben ook de modernisatie van Excel persoonlijk meegemaakt (MS Power BI initiatief). Voor ETL plugin Power Query heb ik de eerste library van herbruikbare functies geschreven; over self-service BI tool PowerPivot heb ik één-op-één kunnen leren van Microsoft’s product manager hiervoor. Inmiddels hebben spreadsheets geen geheimen mee voor me.

% excel/bi
Ik was met Excel begonnen om er beter in te worden dan mijn vader (programmeur) en oom (econoom bij Shell). Sindsdien heb ik ermee het online monitoring programma van de overheid van Singapore opgezet, evenals social media marketing analyses voor verscheidene multi-nationals in Shanghai. Voor MS Power BI heb ik de eerste library van herbruikbare functies geschreven; over PowerPivot heb ik één-op-één kunnen leren van Microsoft’s product manager.

% dbs
Over de jaren heb ik verder ervaring opgedaan met verscheidene databases, waaronder relationeel (postgres, sql server, mysql), noSQL (mongo), column stores, en graph (neo4j), evenals bijbehorende concepten (star schemas, MDX/pivots, Kimball). Het was tevens o.b.v. SQL dat ik een analyze had gedaan van de fonetieke logica van Chinese karakters. Maar ik wil zelf nog eens een compiler maken om DAX, query taal uit MS Power BI, naar SQL te compileren voor cross-table pivots in big data settings (b.v. Spark) of op GPU dbs.

% front
Ik heb verder enkele jaren ervaring opgebouwd als front-end engineer met een focus op Angular. Hier heb ik bijgedragen aan Pug integratie, het aanpassen van de JavaScript standard library voor in point-free stijl, de eerste type library in TypeScript, functie toolkit Ramda.js (typings herschreven), evenals een stel pull requests aan de TypeScript compiler om de type inference te versterken.

% bigdata?


% ml/stats?

% rest
Ik heb verder ervaring met Python (data analyse o.b.v. Pandas, Scipy, Bokeh; web scraping o.b.v. Scrapy; ML o.b.v. ScikitLearn, PyTorch, TensorFlow), Scala (big data processing o.b.v. Spark/Kafka), R (stock scraping + sentiment analysis, ETL van adem-spectra voor classificatie), Matlab, en Haskell.

% EN:

% intro
I'm a curious developer with experience in business intelligence, big data, front-end engineering, functional programming, compilers, and machine learning. I've also been active in the open-source community (Github), and in my spare time try to learn more about building neural networks.

% excel/bi
I started Excel to get better than my father (programmer) and uncle (economist at Shell). I've since used it to set up the online monitoring program for the government of Singapore, among social media marketing analyses for various multi-nationals in Shanghai. For MS Power BI I wrote the first library of reusable functions; for PowerPivot I got to learn 1-on-1 from Microsoft's product manager.

% dbs
Over the years I've gotten experience with different database models, including relational (postgres, sql server, mysql), noSQL (mongo), column stores, and graph (neo4j), as well as related concepts (star schemas, MDX/pivots, Kimball). It was also using SQL that I analyzed the phonetic logic of Chinese characters. But my pet idea in this area is to make a compiler for DAX, query language from MS Power BI, to SQL, to enable cross-table pivots in big data settings (e.g. Spark) or on GPU dbs.

% front
I've also gained a couple years of experience as a front-end engineer, with a focus on Angular. Here I contributed to Pug integration, enabling to use the JS standard library in point-free style, wrote the first type library in TypeScript, rewrote typings for functional toolkit Ramda.js, and made a dozen pull requests to the TypeScript compiler to improve type inference.

% rest
I also have experience with Python (data analysis using Pandas, Scipy, Bokeh; web scraping using Scrapy; ML using ScikitLearn, PyTorch, TensorFlow), Scala (big data processing using Spark/Kafka), R (stock scraping + sentiment analysis, ETL on medical data for classification), Matlab, and Haskell.
